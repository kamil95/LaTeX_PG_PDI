  
% WZOR DODAWANIA OBRAZKA - w nawiasach kwadratowych, np. [ht] formatowanie położenia

\begin{figure}
  \centering
  \includegraphics[height=3cm]{Karulauk.jpg}
  \caption{Obraz numer 2}
  \label{figure:example1}
\end{figure}

\begin{figure}[ht]
  \centering
  \includegraphics[height=3cm]{Karulauk.jpg}
  \caption{Obraz numer 10}
  \label{figure10}
\end{figure}

% WZOR DODAWANIA ROWNAN:

\begin{equation} \label{mc2}
E = mc^2
\end{equation}
\begin{equation} \label{beta}
\beta = \omega\sqrt{\mu\varepsilon}
\end{equation}
Z równania \eqref{mc2} i \eqref{beta}, po kilku prostych przekształceniach, otrzymujemy równanie \eqref{kkk}.
\begin{equation} \label{kkk}
\beta = \frac{\omega}{c}
\end{equation}

\begin{equation} \label{test}
\boldmath
z = \overbrace{
    \underbrace{x}_\text{real} + i
    \underbrace{y}_\text{imaginary}
    }^\text{complex number}
\end{equation}